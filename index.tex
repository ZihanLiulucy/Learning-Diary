% Options for packages loaded elsewhere
\PassOptionsToPackage{unicode}{hyperref}
\PassOptionsToPackage{hyphens}{url}
\PassOptionsToPackage{dvipsnames,svgnames,x11names}{xcolor}
%
\documentclass[
  letterpaper,
  DIV=11,
  numbers=noendperiod]{scrreprt}

\usepackage{amsmath,amssymb}
\usepackage{iftex}
\ifPDFTeX
  \usepackage[T1]{fontenc}
  \usepackage[utf8]{inputenc}
  \usepackage{textcomp} % provide euro and other symbols
\else % if luatex or xetex
  \usepackage{unicode-math}
  \defaultfontfeatures{Scale=MatchLowercase}
  \defaultfontfeatures[\rmfamily]{Ligatures=TeX,Scale=1}
\fi
\usepackage{lmodern}
\ifPDFTeX\else  
    % xetex/luatex font selection
\fi
% Use upquote if available, for straight quotes in verbatim environments
\IfFileExists{upquote.sty}{\usepackage{upquote}}{}
\IfFileExists{microtype.sty}{% use microtype if available
  \usepackage[]{microtype}
  \UseMicrotypeSet[protrusion]{basicmath} % disable protrusion for tt fonts
}{}
\makeatletter
\@ifundefined{KOMAClassName}{% if non-KOMA class
  \IfFileExists{parskip.sty}{%
    \usepackage{parskip}
  }{% else
    \setlength{\parindent}{0pt}
    \setlength{\parskip}{6pt plus 2pt minus 1pt}}
}{% if KOMA class
  \KOMAoptions{parskip=half}}
\makeatother
\usepackage{xcolor}
\setlength{\emergencystretch}{3em} % prevent overfull lines
\setcounter{secnumdepth}{5}
% Make \paragraph and \subparagraph free-standing
\ifx\paragraph\undefined\else
  \let\oldparagraph\paragraph
  \renewcommand{\paragraph}[1]{\oldparagraph{#1}\mbox{}}
\fi
\ifx\subparagraph\undefined\else
  \let\oldsubparagraph\subparagraph
  \renewcommand{\subparagraph}[1]{\oldsubparagraph{#1}\mbox{}}
\fi

\usepackage{color}
\usepackage{fancyvrb}
\newcommand{\VerbBar}{|}
\newcommand{\VERB}{\Verb[commandchars=\\\{\}]}
\DefineVerbatimEnvironment{Highlighting}{Verbatim}{commandchars=\\\{\}}
% Add ',fontsize=\small' for more characters per line
\usepackage{framed}
\definecolor{shadecolor}{RGB}{241,243,245}
\newenvironment{Shaded}{\begin{snugshade}}{\end{snugshade}}
\newcommand{\AlertTok}[1]{\textcolor[rgb]{0.68,0.00,0.00}{#1}}
\newcommand{\AnnotationTok}[1]{\textcolor[rgb]{0.37,0.37,0.37}{#1}}
\newcommand{\AttributeTok}[1]{\textcolor[rgb]{0.40,0.45,0.13}{#1}}
\newcommand{\BaseNTok}[1]{\textcolor[rgb]{0.68,0.00,0.00}{#1}}
\newcommand{\BuiltInTok}[1]{\textcolor[rgb]{0.00,0.23,0.31}{#1}}
\newcommand{\CharTok}[1]{\textcolor[rgb]{0.13,0.47,0.30}{#1}}
\newcommand{\CommentTok}[1]{\textcolor[rgb]{0.37,0.37,0.37}{#1}}
\newcommand{\CommentVarTok}[1]{\textcolor[rgb]{0.37,0.37,0.37}{\textit{#1}}}
\newcommand{\ConstantTok}[1]{\textcolor[rgb]{0.56,0.35,0.01}{#1}}
\newcommand{\ControlFlowTok}[1]{\textcolor[rgb]{0.00,0.23,0.31}{#1}}
\newcommand{\DataTypeTok}[1]{\textcolor[rgb]{0.68,0.00,0.00}{#1}}
\newcommand{\DecValTok}[1]{\textcolor[rgb]{0.68,0.00,0.00}{#1}}
\newcommand{\DocumentationTok}[1]{\textcolor[rgb]{0.37,0.37,0.37}{\textit{#1}}}
\newcommand{\ErrorTok}[1]{\textcolor[rgb]{0.68,0.00,0.00}{#1}}
\newcommand{\ExtensionTok}[1]{\textcolor[rgb]{0.00,0.23,0.31}{#1}}
\newcommand{\FloatTok}[1]{\textcolor[rgb]{0.68,0.00,0.00}{#1}}
\newcommand{\FunctionTok}[1]{\textcolor[rgb]{0.28,0.35,0.67}{#1}}
\newcommand{\ImportTok}[1]{\textcolor[rgb]{0.00,0.46,0.62}{#1}}
\newcommand{\InformationTok}[1]{\textcolor[rgb]{0.37,0.37,0.37}{#1}}
\newcommand{\KeywordTok}[1]{\textcolor[rgb]{0.00,0.23,0.31}{#1}}
\newcommand{\NormalTok}[1]{\textcolor[rgb]{0.00,0.23,0.31}{#1}}
\newcommand{\OperatorTok}[1]{\textcolor[rgb]{0.37,0.37,0.37}{#1}}
\newcommand{\OtherTok}[1]{\textcolor[rgb]{0.00,0.23,0.31}{#1}}
\newcommand{\PreprocessorTok}[1]{\textcolor[rgb]{0.68,0.00,0.00}{#1}}
\newcommand{\RegionMarkerTok}[1]{\textcolor[rgb]{0.00,0.23,0.31}{#1}}
\newcommand{\SpecialCharTok}[1]{\textcolor[rgb]{0.37,0.37,0.37}{#1}}
\newcommand{\SpecialStringTok}[1]{\textcolor[rgb]{0.13,0.47,0.30}{#1}}
\newcommand{\StringTok}[1]{\textcolor[rgb]{0.13,0.47,0.30}{#1}}
\newcommand{\VariableTok}[1]{\textcolor[rgb]{0.07,0.07,0.07}{#1}}
\newcommand{\VerbatimStringTok}[1]{\textcolor[rgb]{0.13,0.47,0.30}{#1}}
\newcommand{\WarningTok}[1]{\textcolor[rgb]{0.37,0.37,0.37}{\textit{#1}}}

\providecommand{\tightlist}{%
  \setlength{\itemsep}{0pt}\setlength{\parskip}{0pt}}\usepackage{longtable,booktabs,array}
\usepackage{calc} % for calculating minipage widths
% Correct order of tables after \paragraph or \subparagraph
\usepackage{etoolbox}
\makeatletter
\patchcmd\longtable{\par}{\if@noskipsec\mbox{}\fi\par}{}{}
\makeatother
% Allow footnotes in longtable head/foot
\IfFileExists{footnotehyper.sty}{\usepackage{footnotehyper}}{\usepackage{footnote}}
\makesavenoteenv{longtable}
\usepackage{graphicx}
\makeatletter
\def\maxwidth{\ifdim\Gin@nat@width>\linewidth\linewidth\else\Gin@nat@width\fi}
\def\maxheight{\ifdim\Gin@nat@height>\textheight\textheight\else\Gin@nat@height\fi}
\makeatother
% Scale images if necessary, so that they will not overflow the page
% margins by default, and it is still possible to overwrite the defaults
% using explicit options in \includegraphics[width, height, ...]{}
\setkeys{Gin}{width=\maxwidth,height=\maxheight,keepaspectratio}
% Set default figure placement to htbp
\makeatletter
\def\fps@figure{htbp}
\makeatother
% definitions for citeproc citations
\NewDocumentCommand\citeproctext{}{}
\NewDocumentCommand\citeproc{mm}{%
  \begingroup\def\citeproctext{#2}\cite{#1}\endgroup}
\makeatletter
 % allow citations to break across lines
 \let\@cite@ofmt\@firstofone
 % avoid brackets around text for \cite:
 \def\@biblabel#1{}
 \def\@cite#1#2{{#1\if@tempswa , #2\fi}}
\makeatother
\newlength{\cslhangindent}
\setlength{\cslhangindent}{1.5em}
\newlength{\csllabelwidth}
\setlength{\csllabelwidth}{3em}
\newenvironment{CSLReferences}[2] % #1 hanging-indent, #2 entry-spacing
 {\begin{list}{}{%
  \setlength{\itemindent}{0pt}
  \setlength{\leftmargin}{0pt}
  \setlength{\parsep}{0pt}
  % turn on hanging indent if param 1 is 1
  \ifodd #1
   \setlength{\leftmargin}{\cslhangindent}
   \setlength{\itemindent}{-1\cslhangindent}
  \fi
  % set entry spacing
  \setlength{\itemsep}{#2\baselineskip}}}
 {\end{list}}
\usepackage{calc}
\newcommand{\CSLBlock}[1]{\hfill\break\parbox[t]{\linewidth}{\strut\ignorespaces#1\strut}}
\newcommand{\CSLLeftMargin}[1]{\parbox[t]{\csllabelwidth}{\strut#1\strut}}
\newcommand{\CSLRightInline}[1]{\parbox[t]{\linewidth - \csllabelwidth}{\strut#1\strut}}
\newcommand{\CSLIndent}[1]{\hspace{\cslhangindent}#1}

\KOMAoption{captions}{tableheading}
\makeatletter
\@ifpackageloaded{bookmark}{}{\usepackage{bookmark}}
\makeatother
\makeatletter
\@ifpackageloaded{caption}{}{\usepackage{caption}}
\AtBeginDocument{%
\ifdefined\contentsname
  \renewcommand*\contentsname{Table of contents}
\else
  \newcommand\contentsname{Table of contents}
\fi
\ifdefined\listfigurename
  \renewcommand*\listfigurename{List of Figures}
\else
  \newcommand\listfigurename{List of Figures}
\fi
\ifdefined\listtablename
  \renewcommand*\listtablename{List of Tables}
\else
  \newcommand\listtablename{List of Tables}
\fi
\ifdefined\figurename
  \renewcommand*\figurename{Figure}
\else
  \newcommand\figurename{Figure}
\fi
\ifdefined\tablename
  \renewcommand*\tablename{Table}
\else
  \newcommand\tablename{Table}
\fi
}
\@ifpackageloaded{float}{}{\usepackage{float}}
\floatstyle{ruled}
\@ifundefined{c@chapter}{\newfloat{codelisting}{h}{lop}}{\newfloat{codelisting}{h}{lop}[chapter]}
\floatname{codelisting}{Listing}
\newcommand*\listoflistings{\listof{codelisting}{List of Listings}}
\makeatother
\makeatletter
\makeatother
\makeatletter
\@ifpackageloaded{caption}{}{\usepackage{caption}}
\@ifpackageloaded{subcaption}{}{\usepackage{subcaption}}
\makeatother
\ifLuaTeX
  \usepackage{selnolig}  % disable illegal ligatures
\fi
\usepackage{bookmark}

\IfFileExists{xurl.sty}{\usepackage{xurl}}{} % add URL line breaks if available
\urlstyle{same} % disable monospaced font for URLs
\hypersetup{
  pdftitle={learning diary},
  pdfauthor={Zihan Liu},
  colorlinks=true,
  linkcolor={blue},
  filecolor={Maroon},
  citecolor={Blue},
  urlcolor={Blue},
  pdfcreator={LaTeX via pandoc}}

\title{learning diary}
\author{Zihan Liu}
\date{2024-01-23}

\begin{document}
\maketitle

\renewcommand*\contentsname{Table of contents}
{
\hypersetup{linkcolor=}
\setcounter{tocdepth}{2}
\tableofcontents
}
\bookmarksetup{startatroot}

\chapter*{Preface}\label{preface}
\addcontentsline{toc}{chapter}{Preface}

\markboth{Preface}{Preface}

This is a Quarto book.

To learn more about Quarto books visit
\url{https://quarto.org/docs/books}.

\bookmarksetup{startatroot}

\chapter{Introduction}\label{introduction}

This is a book created from markdown and executable code.

See Knuth (1984) for additional discussion of literate programming.

\begin{Shaded}
\begin{Highlighting}[]
\DecValTok{1} \SpecialCharTok{+} \DecValTok{1}
\end{Highlighting}
\end{Shaded}

\begin{verbatim}
[1] 2
\end{verbatim}

\bookmarksetup{startatroot}

\chapter{Week01}\label{week01}

The definition of remote sensing

\bookmarksetup{startatroot}

\chapter{Summary:}\label{summary}

Basic knowledge of remote sensing Remotely sensed images and the
corresponding analytical techniques offer~a comprehensive approach to
observing and monitoring urban environments in real-time through high
spatial-temporal-spectral-resolution data

\section{Passive sensor and active
sensors}\label{passive-sensor-and-active-sensors}

There are passive sensor and active sensors, which the difference is the
passive sensor reflect energy from the sun, but active sensors Actively
emits electormagentic waves and then waits to receive them. Example of
passive sensors: Camera, infrared,Thermometers, human eyes Example of
active sensors: Radar, Sonar, X-ray

\section{Formula}\label{formula}

Λ(wavelength)~=~c (velocity of light)~/~v (frequency)

\section{Scatter in action}\label{scatter-in-action}

On the way prior to hitting the sensor, energy maybe absorbed by the
surface or can be scattered by particles in the atmosphere

Why the sky is blue: blue hues have smaller wavelengths which can
scatter easier.

When the sun's angle changes the blue light scatter doesn't reach our
eyes as the distance is increased , so longer wavelengths like reds and
oranges can be seen as they are the longest wavelengths. We can see the
color since there is atmosphere so molecules scatter the light. The
other colors are scattered so we can only see orange or red color.

\section{Interacting with earth's
surface}\label{interacting-with-earths-surface}

BRDF explains light reflection variations on surfaces, influenced by
observation and illumination angles, highlighting changes in surface
visibility.

SAR data, enhanced by polarization, reveals surface characteristics like
roughness and moisture, with single, dual, and quad polarizations
providing different levels of detail.

\section{four resolutions}\label{four-resolutions}

\subsection{Spatial}\label{spatial}

Spatial resolution: Size of raster cells varies from 10cm to several
kilometers.

\subsection{Spectral}\label{spectral}

refers to detecting different wavelengths across the electromagnetic
spectrum, not just the visible light (red, green, blue). An object's
color depend on which wavelengths they reflect, with others being
absorbed or scattered. Our observation are limited due to the
wavelengths absorbed by water vapour, ozone, and other gases. Spectral
resolution classification is based on the number of observed bands
Measuring spectral reflectance~ isn't limited to remote sensors; it can
also be conducted using `spectroradiometers' in labs or fields,
requiring calibration with a pure white reference panel

\subsection{Temporal}\label{temporal}

Sensor's sensitivity to energy is different, with higher resolution
offering more detail (8 bit = 256 values, 4 bit = 16 values).

\subsection{Radiometric}\label{radiometric}

Sensor revisit time, with lower resolution indicating larger pixel size.
Fluorescence in remote sensing is used to identify materials and assess
conditions based on wavelength emissions following radiation exposure.

\section{Biases}\label{biases}

In the UK, cloud cover and atmospheric constituents like water vapor and
carbon dioxide can significantly impact remote sensing data. These
factors obstruct parts of the electromagnetic spectrum, preventing
certain wavelengths from reaching the Earth's surface or sensors. This
interference distorts accurate observations and analysis, leading to
challenges in capturing clear remote sensing imagery. Thus, atmospheric
conditions and clouds are key considerations in remote sensing
applications in the UK.

\bookmarksetup{startatroot}

\chapter{Application:}\label{application}

The International Archives of the Photogrammetry, Remote Sensing and
Spatial Information Sciences~(ISPRS Archives) is the series of
peer-reviewed proceedings published by the International Society of
Photogrammetry and Remote Sensing (ISPRS). I am interested in disaster
environment and climate, so I picked Gi4DM 2020 -- 13th GeoInformation
for Disaster Management conference and did analysis.

Across the world, nature-triggered disasters fuelled by climate change
are worsening. Some two billion people have been affected by the
consequences of natural hazards over the last ten years, 95\% of which
were weather-related (such as floods and windstorms).
https://isprs-archives.copernicus.org/articles/XLIV-3-W1-2020/1/2020/

Australia frequently experiences extended periods of severe droughts
which have a significant negative impact on populations and economy.
Drought indicators selected to compute drought hazard -- the
Standardised Precipitation Index (SPI), the Vegetation Health Index
(VHI) and Soil Moisture -- were obtained through the World
Meteorological Organization (WMO) Space-based Weather and Climate
Extremes Monitoring (SWCEM) international initiative (Name 2020).

https://isprs-archives.copernicus.org/articles/XLIV-3-W1-2020/139/2020/

original

Understanding habitat dynamics in space and time is crucial for
assessing the protection measures' effectiveness and supporting the
definition of sustainable management practices (Orsenigo et
al.~Citation2018). Earth observation by satellite remote sensing (RS)
offers many possibilities for cost-effective, timely, and reproducible
vegetation analysis (Royimani, Mutanga, and Dube~Citation2021;
Vizzari~Citation2022). Grassland remote sensing research may
substantially support ecological studies because it utilizes
multi-platform, multi-sensor, and multi-temporal satellite remote
sensing data sources and ground observation data, through remote sensing
inversion and data assimilation (Li et al.~Citation2021).

~Parracciani,~Daniela~Gigante,~Onisimo~Mutanga,~Stefania~Bonafoni~\&~Marco~Vizzari~(2024)~Land
cover changes in grassland landscapes: combining enhanced Landsat data
composition, LandTrendr, and machine learning classification in google
earth engine with MLP-ANN scenario forecasting,~GIScience \& Remote
Sensing,~61:1,~DOI:~10.1080/15481603.2024.2302221 Critically analyzing
both articles, the integration of remote sensing in disaster assessment
and management is evident. In the articles, remote sensing data,
particularly from the WMO initiative, is utilized for monitoring drought
indicators. The second paragraph expands on the broader applications of
remote sensing, particularly in assessing land cover changes and
vegetation dynamics. It mentioned thye temporal analysis from remote
sensing, which gave me a more comprehensive knowledge about it. To
manage disasters and reduce their bad effects, we can use data from
remote sensing. An example of this is land cover maps, they can help us
understand how vegetation and land use have changed over time as well as
see changes in how land is used.

\section{Reflection:}\label{reflection}

Reflecting on my journey through learning remote sensing, I noticed the
depth and of this field. Initially, I thought it as a straightforward
method of observing the Earth from space, but have understood the
complex of its technologies and principles. Understanding the difference
between passive and active sensors was a key learning moment for me. And
the example of passive sensor of human eye helped me to leern the
knowledge. Sensors could either rely on the Earth's natural energy or
create their own to study the environment. It's interesting how these
sensors, through different mechanisms, contribute to a comprehensive
view of our planet's surface and atmosphere. The scientific foundation,
particularly the principles of light wavelength, speed, and frequency
has remind me the basic knowledge from a-level physics. The explanation
of why the sky is blue, based on the scattering of light, provided a
down-to-earth example of how remote sensing blends physics with
environmental science, making the abstract more understandable. Facing
the challenges of atmospheric interference in remote sensing, especially
in regions like the UK where there are always cloud cover, emphasized
the complexities and limitations. It made me realize the importance of
keeping low bias when practice. What I'm most interested in this
learning journey is how remote sensing serves as a critical tool in
disaster management and environmental monitoring. The ability to use
this technology to track spatial and temporal changes, assess damage,
and support sustainable practices brings a hopeful perspective on
addressing global climate and environmental challenges.

\section{\texorpdfstring{\textbf{References}}{References}}\label{references}

\bookmarksetup{startatroot}

\chapter{Data Fusion: Principles and
Methods}\label{data-fusion-principles-and-methods}

\section{Introduction}\label{introduction-1}

Data Fusion produce more accurate, complete, and reliable information

\section{1. Why we need Data Fusion}\label{why-we-need-data-fusion}

Data fusion in remote sensing refers to the integration of data from
multiple sources or sensors to obtain a more comprehensive understanding
of the Earth's surface or atmosphere

It can enhance information content: Different sensors capture different
types of information on Earth's surface. By combining data, it can
provide a more complete picture of environmental phenomena such as land
cover, vegetation health, or atmospheric conditions.

It can also improve accuracy: reduce errors and uncertainties in remote
sensing observations. By cross-validating measurements from different
sensors or platforms, data fusion techniques can improve the accuracy of
derived products such as maps, models, or environmental indicators.

In addition, data fusion can proide complementary information. By fusing
data from multiple sources, it's possible to leverage the strengths of
each technique and extract more valuable insights than would be possible
with any single source alone.

\section{2. Data Fusion}\label{data-fusion}

Data fusion in remote sensing images typically follows a two-step
process:

\subsection{Geometric correction solution
modelling}\label{geometric-correction-solution-modelling}

x=a0+a1xi+a2yi+ϵi is the formula to regular the biased x. The goal is to
match the distorted image with the gold standard image\ldots.so we want
the pixels to line up We sum up (observed - predicted (the
residual))\^{}2 and calculate the square root of it to calculate the
RMSE, that is to rotate the input grid to a regular

\subsection{Atmospheric correction}\label{atmospheric-correction}

To remove the haze, we can reduce the contrast of the image. Atmospheric
correlation facilitates the derivation of precise surface reflectance
measurements, which play a crucial role in various remote sensing tasks
including land cover classification, vegetation assessment, and water
quality analysis. \#\#\# Empirical Line Correction

Reflectance (field spectrum) = gain * radiance (input data)
Orthorectification correction aim to removing distortions

\section{3. Remote sensing jargon}\label{remote-sensing-jargon}

Radiance: Radiance refers to the amount of electromagnetic radiation
emitted or reflected by a surface in a particular direction. It is
typically measured in watts per square meter per steradian (W/(m²·sr)).

Irradiance: Irradiance refers to the amount of electromagnetic radiation
incident on a surface per unit area. Specifically, downwelling
irradiance refers to the radiation reaching the Earth's surface from the
sun.

Reflectance: Reflectance is a fundamental property of materials that
describes the proportion of incident radiation that is reflected by a
surface.

\section{4. Application(unfinished)}\label{applicationunfinished}

Literature 1: One of the objectives of remote sensing is to go beyond
simple visual interpretation in order to provide the user with
quantitative information for producing documents that conform to
cartographic standards and for deriving digital data files compatible
with geographical information systems (GIS). In this framework, rigorous
geometrical correction is essential. Error sources which introduce
geometrical image distortions are related to the platform vector
(attitude, altitude, speed), the sensor (distortions, oblique viewing),
and to the earth (rotation, earth curvature, ellipsoid, relief).~Wald's
definition of data fusion in remote sensing emphasizes integrating data
from diverse sources to improve overall information quality, aligning
with the specific needs of remote sensing applications. Ruser and Puente
Leon's definition highlights the process of combining data from various
sensors to enhance understanding of physical phenomena, emphasizing
dynamic analysis for insights into both static and evolving scenarios
relevant to remote sensing. Li et al.'s definition underscores the aim
of enhancing signal quality and reliability through the fusion of data
from multiple sensors, in line with the objectives of remote sensing
data processing. Overall, these definitions collectively stress the
importance of integrating data from different sources to enhance
knowledge and achieve better outcomes in remote sensing endeavors.

\bookmarksetup{startatroot}

\chapter*{References}\label{references-1}
\addcontentsline{toc}{chapter}{References}

\markboth{References}{References}

\phantomsection\label{refs}
\begin{CSLReferences}{1}{0}
\bibitem[\citeproctext]{ref-knuth84}
Knuth, Donald E. 1984. {``Literate Programming.''} \emph{Comput. J.} 27
(2): 97--111. \url{https://doi.org/10.1093/comjnl/27.2.97}.

\bibitem[\citeproctext]{ref-YourLabel}
Name, Author(s). 2020. {``Title of the Article.''} \emph{ISPRS Archives}
XLIV-3-W1-2020: 139.
\url{https://isprs-archives.copernicus.org/articles/XLIV-3-W1-2020/139/2020/}.

\end{CSLReferences}



\end{document}

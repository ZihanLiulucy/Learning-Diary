% Options for packages loaded elsewhere
\PassOptionsToPackage{unicode}{hyperref}
\PassOptionsToPackage{hyphens}{url}
\PassOptionsToPackage{dvipsnames,svgnames,x11names}{xcolor}
%
\documentclass[
  letterpaper,
  DIV=11,
  numbers=noendperiod]{scrreprt}

\usepackage{amsmath,amssymb}
\usepackage{iftex}
\ifPDFTeX
  \usepackage[T1]{fontenc}
  \usepackage[utf8]{inputenc}
  \usepackage{textcomp} % provide euro and other symbols
\else % if luatex or xetex
  \usepackage{unicode-math}
  \defaultfontfeatures{Scale=MatchLowercase}
  \defaultfontfeatures[\rmfamily]{Ligatures=TeX,Scale=1}
\fi
\usepackage{lmodern}
\ifPDFTeX\else  
    % xetex/luatex font selection
\fi
% Use upquote if available, for straight quotes in verbatim environments
\IfFileExists{upquote.sty}{\usepackage{upquote}}{}
\IfFileExists{microtype.sty}{% use microtype if available
  \usepackage[]{microtype}
  \UseMicrotypeSet[protrusion]{basicmath} % disable protrusion for tt fonts
}{}
\makeatletter
\@ifundefined{KOMAClassName}{% if non-KOMA class
  \IfFileExists{parskip.sty}{%
    \usepackage{parskip}
  }{% else
    \setlength{\parindent}{0pt}
    \setlength{\parskip}{6pt plus 2pt minus 1pt}}
}{% if KOMA class
  \KOMAoptions{parskip=half}}
\makeatother
\usepackage{xcolor}
\setlength{\emergencystretch}{3em} % prevent overfull lines
\setcounter{secnumdepth}{5}
% Make \paragraph and \subparagraph free-standing
\ifx\paragraph\undefined\else
  \let\oldparagraph\paragraph
  \renewcommand{\paragraph}[1]{\oldparagraph{#1}\mbox{}}
\fi
\ifx\subparagraph\undefined\else
  \let\oldsubparagraph\subparagraph
  \renewcommand{\subparagraph}[1]{\oldsubparagraph{#1}\mbox{}}
\fi

\usepackage{color}
\usepackage{fancyvrb}
\newcommand{\VerbBar}{|}
\newcommand{\VERB}{\Verb[commandchars=\\\{\}]}
\DefineVerbatimEnvironment{Highlighting}{Verbatim}{commandchars=\\\{\}}
% Add ',fontsize=\small' for more characters per line
\usepackage{framed}
\definecolor{shadecolor}{RGB}{241,243,245}
\newenvironment{Shaded}{\begin{snugshade}}{\end{snugshade}}
\newcommand{\AlertTok}[1]{\textcolor[rgb]{0.68,0.00,0.00}{#1}}
\newcommand{\AnnotationTok}[1]{\textcolor[rgb]{0.37,0.37,0.37}{#1}}
\newcommand{\AttributeTok}[1]{\textcolor[rgb]{0.40,0.45,0.13}{#1}}
\newcommand{\BaseNTok}[1]{\textcolor[rgb]{0.68,0.00,0.00}{#1}}
\newcommand{\BuiltInTok}[1]{\textcolor[rgb]{0.00,0.23,0.31}{#1}}
\newcommand{\CharTok}[1]{\textcolor[rgb]{0.13,0.47,0.30}{#1}}
\newcommand{\CommentTok}[1]{\textcolor[rgb]{0.37,0.37,0.37}{#1}}
\newcommand{\CommentVarTok}[1]{\textcolor[rgb]{0.37,0.37,0.37}{\textit{#1}}}
\newcommand{\ConstantTok}[1]{\textcolor[rgb]{0.56,0.35,0.01}{#1}}
\newcommand{\ControlFlowTok}[1]{\textcolor[rgb]{0.00,0.23,0.31}{#1}}
\newcommand{\DataTypeTok}[1]{\textcolor[rgb]{0.68,0.00,0.00}{#1}}
\newcommand{\DecValTok}[1]{\textcolor[rgb]{0.68,0.00,0.00}{#1}}
\newcommand{\DocumentationTok}[1]{\textcolor[rgb]{0.37,0.37,0.37}{\textit{#1}}}
\newcommand{\ErrorTok}[1]{\textcolor[rgb]{0.68,0.00,0.00}{#1}}
\newcommand{\ExtensionTok}[1]{\textcolor[rgb]{0.00,0.23,0.31}{#1}}
\newcommand{\FloatTok}[1]{\textcolor[rgb]{0.68,0.00,0.00}{#1}}
\newcommand{\FunctionTok}[1]{\textcolor[rgb]{0.28,0.35,0.67}{#1}}
\newcommand{\ImportTok}[1]{\textcolor[rgb]{0.00,0.46,0.62}{#1}}
\newcommand{\InformationTok}[1]{\textcolor[rgb]{0.37,0.37,0.37}{#1}}
\newcommand{\KeywordTok}[1]{\textcolor[rgb]{0.00,0.23,0.31}{#1}}
\newcommand{\NormalTok}[1]{\textcolor[rgb]{0.00,0.23,0.31}{#1}}
\newcommand{\OperatorTok}[1]{\textcolor[rgb]{0.37,0.37,0.37}{#1}}
\newcommand{\OtherTok}[1]{\textcolor[rgb]{0.00,0.23,0.31}{#1}}
\newcommand{\PreprocessorTok}[1]{\textcolor[rgb]{0.68,0.00,0.00}{#1}}
\newcommand{\RegionMarkerTok}[1]{\textcolor[rgb]{0.00,0.23,0.31}{#1}}
\newcommand{\SpecialCharTok}[1]{\textcolor[rgb]{0.37,0.37,0.37}{#1}}
\newcommand{\SpecialStringTok}[1]{\textcolor[rgb]{0.13,0.47,0.30}{#1}}
\newcommand{\StringTok}[1]{\textcolor[rgb]{0.13,0.47,0.30}{#1}}
\newcommand{\VariableTok}[1]{\textcolor[rgb]{0.07,0.07,0.07}{#1}}
\newcommand{\VerbatimStringTok}[1]{\textcolor[rgb]{0.13,0.47,0.30}{#1}}
\newcommand{\WarningTok}[1]{\textcolor[rgb]{0.37,0.37,0.37}{\textit{#1}}}

\providecommand{\tightlist}{%
  \setlength{\itemsep}{0pt}\setlength{\parskip}{0pt}}\usepackage{longtable,booktabs,array}
\usepackage{calc} % for calculating minipage widths
% Correct order of tables after \paragraph or \subparagraph
\usepackage{etoolbox}
\makeatletter
\patchcmd\longtable{\par}{\if@noskipsec\mbox{}\fi\par}{}{}
\makeatother
% Allow footnotes in longtable head/foot
\IfFileExists{footnotehyper.sty}{\usepackage{footnotehyper}}{\usepackage{footnote}}
\makesavenoteenv{longtable}
\usepackage{graphicx}
\makeatletter
\def\maxwidth{\ifdim\Gin@nat@width>\linewidth\linewidth\else\Gin@nat@width\fi}
\def\maxheight{\ifdim\Gin@nat@height>\textheight\textheight\else\Gin@nat@height\fi}
\makeatother
% Scale images if necessary, so that they will not overflow the page
% margins by default, and it is still possible to overwrite the defaults
% using explicit options in \includegraphics[width, height, ...]{}
\setkeys{Gin}{width=\maxwidth,height=\maxheight,keepaspectratio}
% Set default figure placement to htbp
\makeatletter
\def\fps@figure{htbp}
\makeatother
% definitions for citeproc citations
\NewDocumentCommand\citeproctext{}{}
\NewDocumentCommand\citeproc{mm}{%
  \begingroup\def\citeproctext{#2}\cite{#1}\endgroup}
\makeatletter
 % allow citations to break across lines
 \let\@cite@ofmt\@firstofone
 % avoid brackets around text for \cite:
 \def\@biblabel#1{}
 \def\@cite#1#2{{#1\if@tempswa , #2\fi}}
\makeatother
\newlength{\cslhangindent}
\setlength{\cslhangindent}{1.5em}
\newlength{\csllabelwidth}
\setlength{\csllabelwidth}{3em}
\newenvironment{CSLReferences}[2] % #1 hanging-indent, #2 entry-spacing
 {\begin{list}{}{%
  \setlength{\itemindent}{0pt}
  \setlength{\leftmargin}{0pt}
  \setlength{\parsep}{0pt}
  % turn on hanging indent if param 1 is 1
  \ifodd #1
   \setlength{\leftmargin}{\cslhangindent}
   \setlength{\itemindent}{-1\cslhangindent}
  \fi
  % set entry spacing
  \setlength{\itemsep}{#2\baselineskip}}}
 {\end{list}}
\usepackage{calc}
\newcommand{\CSLBlock}[1]{\hfill\break\parbox[t]{\linewidth}{\strut\ignorespaces#1\strut}}
\newcommand{\CSLLeftMargin}[1]{\parbox[t]{\csllabelwidth}{\strut#1\strut}}
\newcommand{\CSLRightInline}[1]{\parbox[t]{\linewidth - \csllabelwidth}{\strut#1\strut}}
\newcommand{\CSLIndent}[1]{\hspace{\cslhangindent}#1}

\KOMAoption{captions}{tableheading}
\makeatletter
\@ifpackageloaded{bookmark}{}{\usepackage{bookmark}}
\makeatother
\makeatletter
\@ifpackageloaded{caption}{}{\usepackage{caption}}
\AtBeginDocument{%
\ifdefined\contentsname
  \renewcommand*\contentsname{Table of contents}
\else
  \newcommand\contentsname{Table of contents}
\fi
\ifdefined\listfigurename
  \renewcommand*\listfigurename{List of Figures}
\else
  \newcommand\listfigurename{List of Figures}
\fi
\ifdefined\listtablename
  \renewcommand*\listtablename{List of Tables}
\else
  \newcommand\listtablename{List of Tables}
\fi
\ifdefined\figurename
  \renewcommand*\figurename{Figure}
\else
  \newcommand\figurename{Figure}
\fi
\ifdefined\tablename
  \renewcommand*\tablename{Table}
\else
  \newcommand\tablename{Table}
\fi
}
\@ifpackageloaded{float}{}{\usepackage{float}}
\floatstyle{ruled}
\@ifundefined{c@chapter}{\newfloat{codelisting}{h}{lop}}{\newfloat{codelisting}{h}{lop}[chapter]}
\floatname{codelisting}{Listing}
\newcommand*\listoflistings{\listof{codelisting}{List of Listings}}
\makeatother
\makeatletter
\makeatother
\makeatletter
\@ifpackageloaded{caption}{}{\usepackage{caption}}
\@ifpackageloaded{subcaption}{}{\usepackage{subcaption}}
\makeatother
\ifLuaTeX
  \usepackage{selnolig}  % disable illegal ligatures
\fi
\usepackage{bookmark}

\IfFileExists{xurl.sty}{\usepackage{xurl}}{} % add URL line breaks if available
\urlstyle{same} % disable monospaced font for URLs
\hypersetup{
  pdftitle={learning diary},
  pdfauthor={Zihan Liu},
  colorlinks=true,
  linkcolor={blue},
  filecolor={Maroon},
  citecolor={Blue},
  urlcolor={Blue},
  pdfcreator={LaTeX via pandoc}}

\title{learning diary}
\author{Zihan Liu}
\date{2024-01-23}

\begin{document}
\maketitle

\renewcommand*\contentsname{Table of contents}
{
\hypersetup{linkcolor=}
\setcounter{tocdepth}{2}
\tableofcontents
}
\bookmarksetup{startatroot}

\chapter*{Preface}\label{preface}
\addcontentsline{toc}{chapter}{Preface}

\markboth{Preface}{Preface}

This is a Quarto book.

To learn more about Quarto books visit
\url{https://quarto.org/docs/books}.

\bookmarksetup{startatroot}

\chapter{Introduction}\label{introduction}

This is a book created from markdown and executable code.

See Knuth (1984) for additional discussion of literate programming.

\begin{Shaded}
\begin{Highlighting}[]
\DecValTok{1} \SpecialCharTok{+} \DecValTok{1}
\end{Highlighting}
\end{Shaded}

\begin{verbatim}
[1] 2
\end{verbatim}

\bookmarksetup{startatroot}

\chapter{Week01}\label{week01}

The definition of remote sensing

\bookmarksetup{startatroot}

\chapter{Summary:}\label{summary}

Basic knowledge of remote sensing Remotely sensed images and the
corresponding analytical techniques offer~a comprehensive approach to
observing and monitoring urban environments in real-time through high
spatial-temporal-spectral-resolution data

\section{Passive sensor and active
sensors}\label{passive-sensor-and-active-sensors}

There are passive sensor and active sensors, which the difference is the
passive sensor reflect energy from the sun, but active sensors Actively
emits electormagentic waves and then waits to receive them. Example of
passive sensors: Camera, infrared,Thermometers, human eyes Example of
active sensors: Radar, Sonar, X-ray

\section{Formula}\label{formula}

Λ(wavelength)~=~c (velocity of light)~/~v (frequency)

\section{Scatter in action}\label{scatter-in-action}

On the way prior to hitting the sensor, energy maybe absorbed by the
surface or can be scattered by particles in the atmosphere

Why the sky is blue: blue hues have smaller wavelengths which can
scatter easier.

When the sun's angle changes the blue light scatter doesn't reach our
eyes as the distance is increased , so longer wavelengths like reds and
oranges can be seen as they are the longest wavelengths. We can see the
color since there is atmosphere so molecules scatter the light. The
other colors are scattered so we can only see orange or red color.

\section{Interacting with earth's
surface}\label{interacting-with-earths-surface}

BRDF explains light reflection variations on surfaces, influenced by
observation and illumination angles, highlighting changes in surface
visibility.

SAR data, enhanced by polarization, reveals surface characteristics like
roughness and moisture, with single, dual, and quad polarizations
providing different levels of detail.

\section{four resolutions}\label{four-resolutions}

\subsection{Spatial}\label{spatial}

Spatial resolution: Size of raster cells varies from 10cm to several
kilometers.

\subsection{Spectral}\label{spectral}

refers to detecting different wavelengths across the electromagnetic
spectrum, not just the visible light (red, green, blue). An object's
color depend on which wavelengths they reflect, with others being
absorbed or scattered. Our observation are limited due to the
wavelengths absorbed by water vapour, ozone, and other gases. Spectral
resolution classification is based on the number of observed bands
Measuring spectral reflectance~ isn't limited to remote sensors; it can
also be conducted using `spectroradiometers' in labs or fields,
requiring calibration with a pure white reference panel

\subsection{Temporal}\label{temporal}

Sensor's sensitivity to energy is different, with higher resolution
offering more detail (8 bit = 256 values, 4 bit = 16 values).

\subsection{Radiometric}\label{radiometric}

Sensor revisit time, with lower resolution indicating larger pixel size.
Fluorescence in remote sensing is used to identify materials and assess
conditions based on wavelength emissions following radiation exposure.

\section{Biases}\label{biases}

In the UK, cloud cover and atmospheric constituents like water vapor and
carbon dioxide can significantly impact remote sensing data. These
factors obstruct parts of the electromagnetic spectrum, preventing
certain wavelengths from reaching the Earth's surface or sensors. This
interference distorts accurate observations and analysis, leading to
challenges in capturing clear remote sensing imagery. Thus, atmospheric
conditions and clouds are key considerations in remote sensing
applications in the UK.

\bookmarksetup{startatroot}

\chapter{Application:}\label{application}

The International Archives of the Photogrammetry, Remote Sensing and
Spatial Information Sciences~(ISPRS Archives) is the series of
peer-reviewed proceedings published by the International Society of
Photogrammetry and Remote Sensing (ISPRS). I am interested in disaster
environment and climate, so I picked Gi4DM 2020 -- 13th GeoInformation
for Disaster Management conference and did analysis.

Across the world, nature-triggered disasters fuelled by climate change
are worsening. Some two billion people have been affected by the
consequences of natural hazards over the last ten years, 95\% of which
were weather-related (such as floods and windstorms).
https://isprs-archives.copernicus.org/articles/XLIV-3-W1-2020/1/2020/

Australia frequently experiences extended periods of severe droughts
which have a significant negative impact on populations and economy.
Drought indicators selected to compute drought hazard -- the
Standardised Precipitation Index (SPI), the Vegetation Health Index
(VHI) and Soil Moisture -- were obtained through the World
Meteorological Organization (WMO) Space-based Weather and Climate
Extremes Monitoring (SWCEM) international initiative (Name 2020).

https://isprs-archives.copernicus.org/articles/XLIV-3-W1-2020/139/2020/

original

Understanding habitat dynamics in space and time is crucial for
assessing the protection measures' effectiveness and supporting the
definition of sustainable management practices (Orsenigo et
al.~Citation2018). Earth observation by satellite remote sensing (RS)
offers many possibilities for cost-effective, timely, and reproducible
vegetation analysis (Royimani, Mutanga, and Dube~Citation2021;
Vizzari~Citation2022). Grassland remote sensing research may
substantially support ecological studies because it utilizes
multi-platform, multi-sensor, and multi-temporal satellite remote
sensing data sources and ground observation data, through remote sensing
inversion and data assimilation (Li et al.~Citation2021).

~Parracciani,~Daniela~Gigante,~Onisimo~Mutanga,~Stefania~Bonafoni~\&~Marco~Vizzari~(2024)~Land
cover changes in grassland landscapes: combining enhanced Landsat data
composition, LandTrendr, and machine learning classification in google
earth engine with MLP-ANN scenario forecasting,~GIScience \& Remote
Sensing,~61:1,~DOI:~10.1080/15481603.2024.2302221 Critically analyzing
both articles, the integration of remote sensing in disaster assessment
and management is evident. In the articles, remote sensing data,
particularly from the WMO initiative, is utilized for monitoring drought
indicators. The second paragraph expands on the broader applications of
remote sensing, particularly in assessing land cover changes and
vegetation dynamics. It mentioned thye temporal analysis from remote
sensing, which gave me a more comprehensive knowledge about it. To
manage disasters and reduce their bad effects, we can use data from
remote sensing. An example of this is land cover maps, they can help us
understand how vegetation and land use have changed over time as well as
see changes in how land is used.

\section{Reflection:}\label{reflection}

Reflecting on my journey through learning remote sensing, I noticed the
depth and of this field. Initially, I thought it as a straightforward
method of observing the Earth from space, but have understood the
complex of its technologies and principles. Understanding the difference
between passive and active sensors was a key learning moment for me. And
the example of passive sensor of human eye helped me to leern the
knowledge. Sensors could either rely on the Earth's natural energy or
create their own to study the environment. It's interesting how these
sensors, through different mechanisms, contribute to a comprehensive
view of our planet's surface and atmosphere. The scientific foundation,
particularly the principles of light wavelength, speed, and frequency
has remind me the basic knowledge from a-level physics. The explanation
of why the sky is blue, based on the scattering of light, provided a
down-to-earth example of how remote sensing blends physics with
environmental science, making the abstract more understandable. Facing
the challenges of atmospheric interference in remote sensing, especially
in regions like the UK where there are always cloud cover, emphasized
the complexities and limitations. It made me realize the importance of
keeping low bias when practice. What I'm most interested in this
learning journey is how remote sensing serves as a critical tool in
disaster management and environmental monitoring. The ability to use
this technology to track spatial and temporal changes, assess damage,
and support sustainable practices brings a hopeful perspective on
addressing global climate and environmental challenges.

\section{\texorpdfstring{\textbf{References}}{References}}\label{references}

\bookmarksetup{startatroot}

\chapter{Week3}\label{week3}

\section{Introduction}\label{introduction-1}

In this section we learned how to deal with the mistakes and
unclearities in the remote sensed images. They sometimes include lots of
biases and flaws like in UK the atmosphere us always a problem in
reading all the photos.

\section{1. Why we need Data Fusion}\label{why-we-need-data-fusion}

Data fusion in remote sensing refers to the integration of data from
multiple sources or sensors to obtain a more comprehensive understanding
of the Earth's surface or atmosphere

It can Enhance Information Content: Different sensors capture different
types of information on Earth's surface. By combining data, it can
provide a more complete picture of environmental phenomena such as land
cover, vegetation health, or atmospheric conditions.

Geometric correction is the process of correcting the image geometry to
ensure that it accurately represents the Earth's surface as if the image
were captured from directly overhead (nadir point of view). This
correction is necessary because images are often taken at an angle and
from a moving platform.

\section{Data correction}\label{data-correction}

We mainly introduce 4 types of data correction here

\subsection{Geometric correlation}\label{geometric-correlation}

Geometric correction in remote sensing adjusts satellite images for
distortions caused by off-nadir viewing angles, topography, wind, and
Earth's rotation. We take the coordinates and model them to give
geometric transformation coefficients. Ground Control Points (GCPs) are
identified and matched with known points on local maps, other images, or
GPS data to model geometric transformation coefficients. The process
involves linear regression to align distorted coordinates, aiming to
minimize the Root Mean Square Error (RMSE), Jensen suggesting a value of
0.5. Transformation algorithms correct coordinates, ensuring the
rectified image aligns accurately with real-world locations. Resampling
the final raster image involves methods like Nearest Neighbor, Linear,
Cubic, and Cubic spline to ensure data accuracy and alignment.

\subsection{Atmospheric correction}\label{atmospheric-correction}

Atmospheric correction is used to counteract atmospheric scattering and
topographic attenuation. Necessary for biophysical parameters needed,
but not for single image classification or multi-date imagery without
biophysical parameter analysis. Correction methods include relative
approaches and absolute approaches. Relative atmospheric correction
normalizes pixel values within images based on reference points, used to
reduce atmospheric effects like haze. Absolute correction converts
observed data to physical quantities like surface reflectance using
atmospheric models, ensuring comparability across images and time.
Methods like Dark Object Subtraction (DOS) and Pseudo-Invariant Features
(PIFs) are typical in relative correction, while absolute approaches use
models like MODTRAN and tools like FLAASH for precise corrections.

\subsection{Empirical Line Correction}\label{empirical-line-correction}

Reflectance (field spectrum) = gain * radiance (input data)
Orthorectification correction aim to removing distortions

Empirical Methods: Include simple Dark Object Subtraction (DS) and the
more complex FLAASH correction, optimizing images by simulating
atmospheric scattering and absorption.

Absolute Correction: Transforms digital brightness values into scaled
surface reflectance, requiring atmospheric models and local condition
data for accurate Earth surface representation.

Empirical Line Correction: Directly corrects images through field
spectrometer measurements taken concurrently with satellite overpasses,
enhancing the accuracy of reflectance data.

\subsection{Orthorectification
correction}\label{orthorectification-correction}

Orthorectification is a subset of georectification, focusing on removing
distortions to make pixels appear as viewed from nadir. Requires
understanding sensor geometry and an elevation model to correct image
distortions.The aim is to make direct and accurate measurements of
distances, angles, positions, and
areas(https://www.satimagingcorp.com/services/orthorectification/).

\section{Joining data}\label{joining-data}

Remote sensing data is often captured as individual, discrete images,
each covering a specific geographical area like pieces of a larger
puzzle, to get a comprehensive understanding of larger regions or to
analyze spatial relationships and patterns across these areas, we need
to merge them together.

Mosaicking in remote sensing is similar to merging in GIS, where
multiple datasets are joined to create a seamless image, by blends the
edges of images, minimizing the visibility of seamlines between joined
images. A base image and a second image are overlapped (20-30\%) to
ensure continuity. Histogram matching algorithms are used within the
overlap area to align brightness values, aiding in seamless integration
before feathering.

\bookmarksetup{startatroot}

\chapter{Application}\label{application-1}

2.I am interested in atmospheric correction as it is useful especially
in cloud-prone areas like the UK. I did a research about four
atmospheric correction methods. Pseudo-Invariant Features (PIF): Uses
stable ground targets to normalize image data, but can struggle with
sample selection and coverage. Radiometric Control Sets (RCS): Utilizes
bright and dark features in images for normalization, balancing
atmospheric effects across the scene. Cost of Sensor Zenith Angle
(COST): An image-based method that corrects using the sun's zenith angle
to adjust for atmospheric absorption and scattering. Second Simulation
of the Satellite Signal in the Solar Spectrum (6S): A modeling approach
that calculates atmospheric impact on reflectance, requiring detailed
input parameters for precise correction. Each method was evaluated for
its effectiveness in dealing with atmospheric interference in satellite
imagery, particularly focusing on vegetation monitoring. The analysis
included image quality, classification results, and landscape metric
changes, identifying COST, RCS, and 6S are better than PIF as they are
more stable and more accurate.

https://www.asprs.org/wp-content/uploads/pers/2007journal/april/2007\_apr\_361-368.pdf

Another literature mentioned that analysis of remote sensing data
requires not only technical data analysis but also expert judgment. In
the scenario of a flood causing potential chemical spills, remote
sensing is employed to identify hazardous substances like ethylene
glycol leaking into waterways near residential areas. Spectral sensors
from an aerial flyover detect a discolored water plume, and specialists
analyze this data against known chemical signatures to assess the spill.
However, this analysis is fraught with challenges: sensor accuracy can't
be guaranteed, similar substances might confuse the identification, and
unknown chemicals may be present, complicating the analysis. Despite
these uncertainties, experts must quickly make informed decisions using
their judgment. This situation underscores the complexities of using
remote sensing in emergencies, where rapid, accurate decision-making is
crucial despite the inherent limitations of the data and the analysis
process. This exemplifies broader issues in technical data analysis,
particularly under urgent conditions where expert judgment is pivotal
yet often unexamined for reliability.

In summary, atmospheric methods enhance image analysis, crucial for
monitoring environmental changes. However, in emergency situations, the
process may face challenges like sensor reliability and unknown
substances, require expert judgment together with technical analysis.

https://ieeexplore-ieee-org.libproxy.ucl.ac.uk/document/4717831

\bookmarksetup{startatroot}

\chapter{Reflection}\label{reflection-1}

Reflecting on my journey as a new student in remote sensing, I've
learned the complex of data fusion and correction.

Data fusion and correction in remote sensing are essential for accurate
Earth observation. Data fusion integrates information from multiple
sources, providing a more comprehensive and detailed view of the
environment, enhancing the analysis of land cover, vegetation health,
and other phenomena. These processes improve the reliability and
accuracy of remote sensing data, essential for environmental monitoring,
disaster management, and various scientific research applications.

Geometric correction was another thing that I'm interested. Learning
about the process, from identifying Ground Control Points (GCPs) to
applying transformation algorithms to correct distortions, was like
complete a puzzle, once done, the true earth surface just appear.

Atmospheric correction, especially in cloudy areas like the UK, became
my focusing point. The exploration of various correction methods, from
Pseudo-Invariant Features (PIF) to the others, showed the complexity and
necessity.

The challenges and opportunities in remote sensing data analysis were
further highlighted in the application part, particularly in emergency
situations. The reliance on expert judgment, together with technical
data analysis, illustrated the dynamic interrelationship between human
expertise and technological capabilities.

In conclusion, my interest into remote sensing has increased. The blend
of technical knowledge, practical application, and the key role of human
judgment in interpreting complex data has enriched my understanding.

\bookmarksetup{startatroot}

\chapter{Week4}\label{week4}

\bookmarksetup{startatroot}

\chapter{Jakarta city challenge}\label{jakarta-city-challenge}

\section{Flood}\label{flood}

Flood-prone Jakarta is the world's fastest sinking city --- as fast as
10 centimetres per year. In parts of North Jakarta, which is
particularly susceptible to flooding, the ground has sunk 2.5 metres in
10 years. Excessive extraction of groundwater for drinking and
commercial use is largely responsible for this: When water is pumped out
of an underground aquifer, the land above it sinks Source from:
https://www.channelnewsasia.com/cnainsider/why-jakarta-is-world-fastest-sinking-city-floods-climate-change-781491

\section{City sinking}\label{city-sinking}

With global temperatures rising and ice sheets melting, plenty of
coastal cities face a growing risk of flooding due to~sea level
rise.~Few places, however, face challenges like those in front of the
Jakarta metropolitan area. In recent decades, Jakarta flooding problems
have grown even worse, driven partly by widespread pumping of
groundwater that has caused the land to sink, or subside, at rapid
rates. By~some estimates, as much as 40 percent of the city now sits
below sea level. There are signs showing that rainstorms are getting
more intense as the atmosphere heats up, damaging floods have become
commonplace.

The picture below shows The Landsat images above show the evolution of
the city over the past three decades. The widespread replacement of
forests and other vegetation with impervious surfaces in inland areas
along the Ciliwung and Cisadane rivers has reduced how much water the
landscape can absorb, contributing to runoff and flash floods.

\bookmarksetup{startatroot}

\chapter{Policy goal}\label{policy-goal}

In September 2022, the Jakarta Environmental Agency introduced
a~Strategy for Air Pollution Control (SPPU), which included more than 70
action plans to improve air quality.~Focusing on 1) governing air
pollution controls, 2) reducing emissions from mobile sources, and 3)
reducing emissions from stationary sources.

\bookmarksetup{startatroot}

\chapter{The actions we can do}\label{the-actions-we-can-do}

\section{Water management to reduce groundwater
extraction}\label{water-management-to-reduce-groundwater-extraction}

Implementing effective water management to reduce groundwater extraction
can align with sustainable urban development goals. As the widespread
pumping of groundwater is the human factor of sinking and flooding, it
is necessary to pretend residents from doing it and worsen the
situation.

\section{Add more green plants to improve the
environment}\label{add-more-green-plants-to-improve-the-environment}

Investing in green infrastructure, can mitigate sinking and enhance
urban resilience, aligning with global climate action and sustainable
city planning. It is obvious from the picture before that the greenery
has been replaced by city land.

Here are benefits from greenery. rain hits the ground at higher speeds
where there is a lack of tree cover. A canopy of leaves, branches and
trunks slows down the rain before it hits the ground simply by getting
in the way. In addition, root systems help water penetrate deeper into
the soil at a faster rate under and around trees.

https://www.woodlandtrust.org.uk/trees-woods-and-wildlife/british-trees/flooding/

\section{Reduce emission}\label{reduce-emission}

Jakarta was again ranked the~most polluted city~in the world by Swiss
technology company IQAir in 2023. Transport is a very important source
of Jakarta's Pollution while ~Industry and Power Plants Are Also
Contributors。 In September 2022, the Jakarta Environmental Agency
introduced a~Strategy for Air Pollution Control (SPPU), which included
more than 70 action plans to improve air quality.~Focusing on 1)
governing air pollution controls, 2) reducing emissions from mobile
sources, and 3) reducing emissions from stationary sources. Source from:
https://urban-links.org/insight/7-things-to-know-about-jakartas-air-pollution-crisis/

Warmer temperatures caused by pollution could lead to the ground
swelling and expanding upwards by up to 12mm (0.5 inches) and~the ground
could sink downwards, beneath the weight of a building, by as much as
8mm (0.3 inches).

Source from:
https://news.sky.com/story/chicago-underground-climate-change-is-deforming-land-under-buildings-and-things-are-sinking-says-study-12923119

\bookmarksetup{startatroot}

\chapter{To manage the emission using remotely sensed
data}\label{to-manage-the-emission-using-remotely-sensed-data}

\section{NO2 and CO2 monitoring}\label{no2-and-co2-monitoring}

Use data from NO2 from https://energyandcleanair.org/~, there are before
and after maps that we can decide the time period and the highly
polluted areas are shown in red. CO2 can be found from our world in data
and the analysis should be in python.

\section{Vegetation Mapping and
Monitoring}\label{vegetation-mapping-and-monitoring}

To manage the pollution, greenery is an indirect method. Satellite and
aerial imagery provide detailed information on vegetation cover,
allowing for the assessment of the extent and health of greenery across
large areas. This helps in tracking changes over time, such as the
growth or decline of forests, parks, and urban green spaces.

\bookmarksetup{startatroot}

\chapter{Link with global agendas /
goals}\label{link-with-global-agendas-goals}

Sustainable Development Goals (SDGs): Particularly relevant are SDG 11
(Sustainable Cities and Communities), SDG 13 (Climate Action), and SDG
15 (Life on Land). Addressing Jakarta's issues contributes to creating
sustainable and resilient cities, taking urgent action to manage climate
change and its impacts, and managing forests and combating
desertification, halting and reversing land degradation, and halting
biodiversity loss.

New Urban Agenda: This agenda emphasizes the need for cities to be
inclusive, safe, resilient, and sustainable. By addressing its sinking
and flooding issues, Jakarta works towards creating a more inclusive
urban environment, ensuring safety and resilience for all its
inhabitants.

\bookmarksetup{startatroot}

\chapter{Advances of current local, national or global
approaches.}\label{advances-of-current-local-national-or-global-approaches.}

Jakarta may set a practical example for sustainable urban development.
By addressing issues related to flood, land subsidence and sea-level
rise, Jakarta contributes to global efforts in climate change
mitigation, providing a case study for coastal city management under
climate threats. Jakarta's initiatives align with the SDGs,
demonstrating how urban areas can deal with complex challenges like
rapid urbanization, climate change, and greenery lose in an integrated
manner.

\bookmarksetup{startatroot}

\chapter{Reflection}\label{reflection-2}

My research revealed plenty of existing strategies addressing urban
challenges; however, it seemed like many straightforward, sensible
solutions are often overlooked by local governments. For Jakarta, the
issue isn't just about addressing the immediate impacts of these
problems but also about considering long-term, sustainable solutions.

While exploring Jakarta's situation, I noticed a tendency for
short-term, reactive measures rather than long-term systemic changes.
For instance, the focus has often been on building sea walls or
improving drainage, which, although necessary, failed to manage the root
causes such as excessive groundwater extraction and less comprehensive
urban planning.

Jakarta requires comprehensive urban planning that considers
environmental sustainability, infrastructure resilience, and community
safe water.

\bookmarksetup{startatroot}

\chapter*{References}\label{references-1}
\addcontentsline{toc}{chapter}{References}

\markboth{References}{References}

\phantomsection\label{refs}
\begin{CSLReferences}{1}{0}
\bibitem[\citeproctext]{ref-knuth84}
Knuth, Donald E. 1984. {``Literate Programming.''} \emph{Comput. J.} 27
(2): 97--111. \url{https://doi.org/10.1093/comjnl/27.2.97}.

\bibitem[\citeproctext]{ref-YourLabel}
Name, Author(s). 2020. {``Title of the Article.''} \emph{ISPRS Archives}
XLIV-3-W1-2020: 139.
\url{https://isprs-archives.copernicus.org/articles/XLIV-3-W1-2020/139/2020/}.

\end{CSLReferences}



\end{document}
